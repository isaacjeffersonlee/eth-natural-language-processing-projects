\documentclass[a4paper]{article}
%Use package definitions
\usepackage[margin=1.3in]{geometry}
\usepackage{amsmath, amssymb, bm}
\usepackage[utf8]{inputenc}
\usepackage[T1]{fontenc}
\usepackage{cmbright}
\usepackage{textcomp}
\usepackage{amsmath, amssymb}
\usepackage{pdfpages}
\usepackage{centernot}
\usepackage{transparent}
\usepackage{varwidth}
\usepackage{xcolor}
\usepackage[most]{tcolorbox}
\usepackage{fancyhdr}
\usepackage{url}
\usepackage{hyperref}
%Algorithm environments
\usepackage{algorithm}
\usepackage{algpseudocode}

%Space between paragraphs
\setlength{\parskip}{10pt}
%No indentation on the first word of a new paragraph
\setlength{\parindent}{0cm}
%Creating a new color box
\newtcolorbox{blackbox}[1][]{colframe=black,colback=white,sharp corners,center,#1}

%Creating the title
\title{Natural Language Processing: Assignment 3\author{Isaac Jefferson Lee}\date{Email: isalee@student.ethz.ch}}

%Creating the header
\pagestyle{fancy}
\fancyhead[L]{Assignment 2}
\fancyhead[R]{\today}
\setlength{\headheight}{14pt} % Fixes headheight warnings when using cmbright

\begin{document}
\maketitle
%Header on title page
\thispagestyle{fancy} 

\section*{Question 1}

\subsection*{Question 1 :: Part a)}
We are given that:
\[
    a^* := \bigoplus_{n=0}^\infty a^{\otimes n}
.\]
So it follows that:
\[
a \otimes a^* = a \otimes \left( \bigoplus_{n=0}^\infty a^{\otimes n} \right) 
.\]
By the distributive property:
\[
    = \bigoplus_{n=0}^\infty (a \otimes a^{\otimes n}) = \bigoplus_{n=0}^\infty a^{\otimes (n+1)}
.\]
We can re-label our index and clearly we have that:
\[
a \otimes a^* = \bigoplus_{m=1}^\infty a^{\otimes m}
.\]
Then it follows that:
\[
\bm{1} \oplus a \otimes a^* = \bm{1} \oplus \bigoplus_{m=1}^\infty a^{\oplus m}
.\]
\textbf{Note:} We must have that $a^{\otimes 0} = \bm{1}$, (easy to prove
by contradiction), so then we have:
\[
    \bm{1} \oplus \bigoplus_{m=1}^\infty a^{\otimes m} = \bigoplus_{m=0}^\infty a^{\otimes m} =: a^*
.\]
QED.

\subsection*{Question 1 :: Part b)}
So we want some $a^* \in \mathbb{R} \cup \{-\infty\} $ such that:
\begin{align*}
     a^* &=  \bm{1} \oplus a \otimes a^* \\
         &= 0 \oplus_{\log} (a + a^*)\\ 
 &=  \log(1 + \exp(a + a^*))\\ 
 &=  \log(1 + \exp(a)\exp(a^*))\\ 
 &\implies \exp(a^*)=1 + \exp(a)\exp(a^*)\\ 
 &\implies \exp(a^*)(1 - \exp(a))=1\\ 
 &\implies \exp(a^*)=\frac{1}{1 - \exp(a)}\\ 
 &\implies a^* =  \log\left(\frac{1}{1-\exp(a)}\right) \\
 &= -\log(1 - \exp(a))
.
\end{align*}
\textbf{Note:} Clearly this is only defined for $a < 0$.

\subsection*{Question 1 :: Part c)}
As before, we seek some $a^*$ s.t:
\[
a^* = \bm{1} \oplus a \otimes a^*
.\]
Translating this to the expectation semiring, we require some $a^* := \langle a_1^*, a_2^* \rangle$ s.t:
\begin{align*}
     \langle a_1^*, a_2^* \rangle &=  \langle 1, 0 \rangle \oplus (\langle a_1, a_2 \rangle \otimes \langle a_1^*, a_2^* \rangle)\\ 
 &=  \langle 1, 0 \rangle \oplus \langle a_1 * a_1^*, a_1 * a_2^* + a_2 * a_1^* \rangle\\ 
 &=  \langle 1 + a_1 * a_1^*, a_1 * a_2^* + a_2*a_1^* \rangle\\ 
.
\end{align*}
So clearly this first term is just the Kleene Star over the Real Semiring, so it follows that:
\[
a_1^* = \sum_{n=0}^{\infty} a_1^n = \frac{1}{1-a_1}
.\]
(Assuming $\left| a_1 \right|  < 1$ for convergence).
So now we want some $a_2^*$ s.t:
\[
a_2^* = a_1 * a_2^* + a_2 * \frac{1}{1-a_1}
.\]
Therefore we can re-arrange this to get:
\[
a_2^*(1-a_1) = \frac{a_2}{1-a_1}
 \implies a_2^* = \frac{a_2}{(1-a_1)^2}
.\]
Then combining we get:
\[
a^* := \langle a_1^*, a_2^* \rangle = \left\langle \frac{1}{1-a_1}, \frac{a_2}{(1-a_1)^2} \right\rangle
.\]


\subsection*{Question 1 :: Part d)}
\[
\mathcal{W}_{\text{lang}} := \langle 2^{\Sigma^*}, \cup, \otimes, \phi, \{\epsilon\}   \rangle
.\]
Where $A \otimes B := \{a \circ b ~~s.t~~ a\in A, b\in B\} $

First we check that $\langle 2^{\Sigma^*, \cup , \phi} \rangle$ is a commutative monoid.
Clearly from elementary set theory we know that union is both commutative and associative.

Also we know that $A \cup \phi = A = \phi \cup A ~\forall A \in 2^{\Sigma^*}$ so clearly $\phi$ is the left
and right unit and so we have verified that $\langle 2^{\Sigma^*, \cup , \phi} \rangle$ is a commutative monoid.

Next we must check that $\langle 2^{\Sigma^*}, \otimes, \{\epsilon\} \rangle$ is a monoid.
Note that:
\begin{align*}
         A \otimes (B \otimes C) &=  A \otimes \{b \circ c ~~s.t~~ b \in B, c \in C\} \\ 
    &=  \{a \circ b \circ c ~~s.t~~ b \in B, c\in C\}
\end{align*}
since string concatenation is associative.
Then $(A\otimes B) \otimes C$ is equivalent by symmetry and so we get associativity.
Next we see that:
\[
A \otimes \{\epsilon\} := \{a \circ \epsilon ~~s.t~~ a \in A\} = A
.\]
Also:
\[
\{\epsilon\} \otimes A := \{\epsilon \circ a ~~s.t~~ a \in A\} = A
.\]
and so clearly $\{\epsilon\} $ is the left and right unit and thus we have verified that
$\langle 2^{\Sigma^*}, \oplus, \{\epsilon\}  \rangle$ is a monoid.

Next we show that $\otimes$ distributes over $\oplus := \cup $.
Suppose $A, B, C \in 2^{\Sigma^*}$, then:
\begin{align*}
         (A \cup B) \otimes C &=  \{\alpha \circ c ~~s.t~~ \alpha \in A\cup B, c \in C\} \\ 
    &=  \{a\circ c ~~s.t~~ a\in A, c\in C\}  \cup \{b \circ c ~~s.t~~ b\in B, c\in C\} \\ 
    &=  (A \otimes C) \cup (B \otimes C)\\ 
.
\end{align*}
Which proves the first part of distributivity and then it is clear to see by symmetry
that the second part of distributivity also holds.

Finally we must show that $\phi$ is an annihilator for $\otimes$.
Suppose $A \in 2^{\Sigma^*}$ then $\phi \otimes A = \{\gamma \circ a ~~s.t~~ \gamma \in \phi, a \in A\} = \phi$.
Since $~~\exists~ \gamma \in \phi$ can never be true. Similarly for $A \otimes \phi$.
So we have verified all of the semiring axioms.

Now we are interested in finding a Kleene Star for $\mathcal{W}_{\text{lang}}$, i.e we want some $A^* \in 2^{\Sigma^*}$ s.t:
\[
A^* = \{\epsilon\} \cup \{a \circ a' ~~s.t~~ a \in A, a' \in A^*\} 
.\]
Clearly the Kleene Closure $\Sigma^*$ satisfies this.
This simply follows from the recursive definition of $\Sigma^*$ given by:
\[
\Sigma^0 := \{\epsilon\}, ~ ~ ~ \Sigma^1 := \Sigma, ~ ~ ~ ~\forall i \in \mathbb{N}, \Sigma^{i+1} := \{w \circ a ~~s.t~~ w \in \Sigma^{i}, a \in \Sigma\} 
.\]
\[
\Sigma^* := \bigcup_{i \in  \mathbb{N}}  \Sigma^i
.\]

\section*{Question 2}

\subsection*{Question 2 :: Part a)}
Recall the Tropical Semiring: $\langle \mathbb{R}_{\ge 0}, min, +, \infty, 0\rangle$.
Suppose we have some arb. $a \in \mathbb{R}_{\ge 0}$, then $\bm{1} \oplus a := min(0, a) = 0 =: \bm{1}$,
since $a \in R_{\ge 0} \implies a\ge 0$. So clearly we have the 0-closed property.

Recall the Arctic Semiring: $\langle \mathbb{R}_{\le 0}, max, +, -\infty, 0 \rangle$.
Suppose we have some arb. $b \in \mathbb{R}_{\le 0}$, then $\bm{1} \oplus b:= max(0, b) = 0 =: \bm{1}$,
since $b \in \mathbb{R}_{\le 0} \implies b \le 0$. So clearly we have the 0-closed property.

\subsection*{Question 2 :: Part b)}
We proceed by induction on $n$.
When $n=1$ the result is trivial. Now assume true for $n = m \in \mathbb{N}$.
Then when $n = m+1$ we have $M^{m+1} = M \otimes M^m$, so:
\[
    (M^{m+1})_{ij} = \bigoplus_{k=1}^N M_{ik} \otimes (M^m)_{kj} = \bigoplus_{k=1}^N w_{ik} \otimes (M^m)_{kj}
.\]
So from our induction hypothesis we know that $(M^m)_{kj}$ is the sum of paths of length exactly $m$,
starting at $k$ and ending at $j$. Clearly $M_{ik} \otimes (M^m)_{kj}$ represents all possible
paths with a single edge from $i$ to $k$ and then the next edges starting at $k$ and ending at $j$.
So then since we are summing over all possible intermediate $k$, it follows that we consider all possible
paths of length $m+1$ and therefore we have that $(M^{m+1})_{ij}$ is the sum of all possible
paths of length $m+1$ starting at node $i$ and ending at $j$ and we are done.

QED.

\subsection*{Question 2 :: Part c)}
\textbf{Main Idea:} Paths of length $>N-1$ must contain cycles.

Suppose we have some path from $i$ to $j$ of length greater than $N-1$.
It is clear that this path will contain atleast one cycle, since we must visit atleast one node  more than once.
So we can factor the path weight as:
\[
w(\pi) = \underbrace{\left( \bigotimes_{k=1}^p w_{k} \right) }_{:= w_{\alpha}(\pi)}\otimes \underbrace{\left( \bigotimes_{k=1}^q w'_k \right) }_{:= w_{\beta}(\pi)}
.\]
where $p+q = M$ and  $w'_k$ are weights from the edges of $\pi$ that we remove to
make $\pi$ a path from $i$ to $j$ with no-cycles. (Note that for large paths with many cycles there could be many different choices of such weights but this choice
is without loss of generality).

So since the non-removed edges contain no cycles, it follows that $p \le  N-1$.
So using our notation, $w(\pi) = w_{\alpha}(\pi) \otimes w_{\beta}(\pi)$.
It follows by definition of $Z(i, j)$ that since $w_{\alpha}(\pi)$ is a valid path
from $i$ to $j$ then both $w_{\alpha}(\pi)$ and $w_{\alpha}(\pi) \otimes w_{\beta}(\pi)$
will be part of the semiring-plus sum given by (2).
So in the sum we will have:
\[
w_{\alpha}(\pi) \oplus (w_{\alpha}(\pi) \otimes w_{\beta}(\pi))
.\]
By distributivity:
\[
= w_{\alpha}(\pi) \otimes (\bm{1} \oplus w_{\beta}(\pi))
.\]
Then by the 0-closed property:
\[
= w_{\alpha}(\pi) \otimes (\bm{1}) = w_{\alpha}(\pi)
.\]
and we can clearly do this for any path of length $> N-1$
and so we have shown that the sum only depends on paths of length  $\le  N-1$.
QED.

\subsection*{Question 2 :: Part d)}
Recall: $M^*$ must satisify:
\[
M^* = \bm{1} \oplus M \oplus M^*
.\]
So defining $M^* := \bigoplus_{n=0}^{N-1} M^n$
we must verify the above.
\[
    M \otimes M^* = M \otimes \bigoplus_{n=0}^{N-1} M^n
.\]
By distributivity:
\[
    = \bigoplus_{n=0}^{N-1} M^{n+1}
.\]
Re-labelling we get:
\[
= \bigoplus_{r=1}^N M^r
.\]
\[
\implies \bm{1} \oplus M \otimes M^* = \bigoplus_{r=0}^N M^r
.\]
since we must have that $M^0 = \bm{1}$.
Then using part b) we know that $M^N$ encodes the paths of length $N$ 
and by part c) we showed that the sum is independent of paths longer than $N-1$ 
and so we see that $M^N$ does not contribute to the sum and we get:
 \[
\bigoplus_{r=0}^N M^r = \bigoplus_{r=0}^{N-1} M^r =: M^*
.\]
and we are done. QED.

\subsection*{Question 2 :: Part e)}
\begin{algorithm}
\caption{Naive $M^*$}
\begin{algorithmic}
    \State  $\mathrm{def~m\_star} (M)$
        \State $M^* \gets \bm{1}$
        \State $\text{prod} \gets \bm{1}$
        \For{$n \in \{1, ..., N-1\} $}
            \State $\text{prod} \gets \text{prod} \otimes M$
            \State $M^* \gets M^* \oplus \text{prod}$
        \EndFor

\Return{$M^*$}

\end{algorithmic}
\end{algorithm}

The for loop gives:
\[
\bm{0} \oplus (\bm{1} \otimes M) \oplus (\bm{1} \otimes M \otimes M) \oplus ... \oplus (\bm{1} \otimes M \otimes ... \otimes M)
.\]
So clearly by definition of $M^n$ we are computing $\bigoplus_{n=0}^{N-1}$ as required.
If we assume that matrix multiplication has runtime complexity $O(N^3)$ using our naive algorithm
we do $N-1$ matrix multiplications therefore we have a worst case runtime complexity:
\[
O(N * N^3) = O(N^4)
.\]

\subsection*{Question 2 :: Part f)}
 \[
\text{0-closed } \implies \bm{1} \oplus \bm{1} = \bm{1} \implies a \otimes (\bm{1} \oplus \bm{1}) = a\otimes \bm{1} = a
.\]
And by distributivity:
\[
a \otimes (\bm{1} \oplus \bm{1}) = a \oplus a \implies a \oplus a = a
.\]
QED.

\subsection*{Question 2 :: Part g)}
By the binomial expansion we get:
\[
    (I \oplus M)^K = \bigoplus_{n=0}^{K} \begin{pmatrix}K\\n\end{pmatrix} I^{K-n} M^n
.\]
By the identity property
\[
= \bigoplus_{n=0}^K \begin{pmatrix}K\\n\end{pmatrix} M^n
.\]
Clearly by idempotency, for any natural number, $\alpha \in \mathbb{N}$:
\[
    \alpha M^n :=  \underbrace{M^n \oplus M^n \oplus ... \oplus M^n}_{\alpha\text{ times}} = M^n
.\]
\[
\implies \bigoplus_{n=0}^K  \begin{pmatrix}K\\n\end{pmatrix} M^n = \bigoplus_{n=0}^K M^n
\]
and we are done. ED.


\subsection*{Question 2 :: Part h)}
Recall that $M^n := \underbrace{M \otimes M \otimes ... \otimes M}_{n\text{ times}}$
so in order for:
\[
    M^n = \bigotimes_{k=0}^{\left\lfloor \log_2(n) \right\rfloor} M^{\alpha_k 2^k}
.\]
to hold, we need to choose $\alpha_k$ such that:
\[
\sum_{k=0}^{\left\lfloor \log_2(n) \right\rfloor} \alpha_k 2^k = n
.\]
clearly we can represent $n$ in binary and therefore such $\alpha_k$ exist.
($\alpha_k$ is given by the $k^{\text{th}}$ digit of the binary representation of $n$).
\[
    (3) + (g) \implies M^* = (I \oplus M)^{N-1}
.\]
\[
    (4) \implies (I \oplus M)^{N-1} = \bigotimes_{k=0}^{\left\lfloor \log_2(N-1) \right\rfloor} (I \oplus M)^{\alpha_k 2^k}
.\]

\begin{algorithm}
\caption{Faster $M^*$}
\begin{algorithmic}
    \State  $\mathrm{def~m\_star\_faster} (M)$
        \State $\bm{\alpha} \gets \text{int\_to\_binary}(N-1)$
        \State $M^* \gets \bm{1}$
        \State $\text{prod} \gets \bm{1}$
        \For{$k \in \{0, ..., \left\lfloor \log_2(N-1) \right\rfloor\} $}
            \If{$\alpha_k == 1$}
                \State $\text{prod} \gets \text{prod} \otimes (I \oplus M)$
                \State $M^* \gets M^* \otimes \text{prod}$
            \EndIf
        \EndFor

\Return{$M^*$}

\end{algorithmic}
\end{algorithm}

We do two matrix multiplications, of the order $\log(N)$ times, so we have overall
runtime complexity $O(\log(N) N^3)$.

\subsection*{Question 2 :: Part i)}
\textbf{Note:} If $A \in \mathbb{R}^{n\times m}$ is a matrix, then the singular values of A are given
by the square roots of the eigenvalues of $A^T A$.

\[
||A||_2 := \sup_{\bm{x} \neq \bm{0}} \frac{||A \bm{x}||_2}{||\bm{x}||_2} = \sup_{||\bm{x}||_2 = 1} ||A \bm{x}||_2
.\]
By singular value composition we have that:
\[
\sup_{||x||_2 = 1} ||A \bm{x}||_2 = \sup_{||\bm{x}||_2 = 1} ||U \Sigma V^T \bm{x}||_2 =
\sup_{||\bm{x}||_2=1} || \Sigma V^T \bm{x}||_2
.\]
Where the last equality follows from the fact that $U$ is unitary.

Define $\bm{y} := V^T \bm{x}$ and then it follows that $||\bm{y}||_2=1$ since $V$ is also unitary.
So we have that:
\[
    \sup_{||\bm{x}||_2=1} ||\Sigma V^T \bm{x}||_2 = \sup_{||\bm{y}||_2=1} ||\Sigma \bm{y}||_2
.\]
Now since we know that $\Sigma_{ij} = \sigma_i$ whenever $i = j$ and  $0$ otherwise,
then and we also defined  $\Sigma$ s.t $\sigma_1 \ge \sigma_2 \ge ... \ge \sigma_m$,
then we see that the max is attained when $\bm{y} = \bm{e}_1$.
This is easy to see from the fact that:
\[
\sqrt{\sum_{i=1}^{m} (\sigma_i y_i)^2}  \le 
\sqrt{\sum_{i=1}^{m} \sigma_i^2 y_i^2} \le 
\sqrt{\sum_{i=1}^{m} \sigma_{\text{max}}^2 y_i^2} \le 
 \sigma_{\text{max}}  \sqrt{\sum_{i=1}^{m} y_i^2} =
\sigma_{\text{max}}
.\]
and we see that $||A||_2 = \sigma_1 = \sigma_{\text{max}}(A)$ 

QED.

\subsection*{Question 2 :: Part j)}
$A^* = \sum_{n=0}^{\infty} A^n$ so it follows that:
\begin{align*}
    ||A^* - \sum_{n=0}^{k} A^n||_2 &=  ||\sum_{n=0}^{\infty} A^n - \sum_{n=0}^{k} A^n||_2\\ 
 &=  ||\sum_{n=k+1}^{\infty} A^n + \sum_{n=0}^{k} A^n - \sum_{n=0}^{k} A^n||_2\\ 
 &=  ||\sum_{n=k+1}^{\infty} A^n||_2\\ 
 &=  ||\lim_{m \to \infty} \sum_{n=k+1}^{m} A^n||_2\\ 
 &=  \lim_{m \to \infty} ||\sum_{n=k+1}^{m} A^n||_2\\ 
.
\end{align*}
By the triangle inequality:
\[
\le \lim_{m \to \infty} \sum_{n=k+1}^{m} ||A^n||_2
.\]
By Cauchy-Schwartz for the operator norm:
\[
\le \lim_{m \to \infty} \sum_{n=k+1}^{m} ||A||_2^n = \lim_{m \to \infty} \sum_{n=k+1}^{m} \sigma_{\text{max}}(A)^n
.\]
So for any fixed $k$, we need this error to converge, so by the root test for infinite
series, if we define $a_n := \sigma_{\text{max}}(A)^n$,
then:
\[
L := \lim_{n \to \infty} \left| a_n \right|^{\frac{1}{n}} = \lim_{n \to \infty} \sigma_{\text{max}}(A) = \sigma_{\text{max}}(A)
.\]
So it follow that for convergence we require $\sigma_{\text{max}}(A) < 1$.
Then we get:
\begin{align*}
     \sum_{n=k+1}^{\infty} \sigma_{	{max}}^n &=  \sum_{n=0}^{\infty} \sigma_{	{max}}^{n} - \sum_{n=0}^{k} \sigma_{	{max}}^{n}\\ 
 &=  \frac{1}{1-\sigma_{	{max}}} - \frac{1 - \sigma_{	{max}}^{k+1}}{1 - \sigma_{	{max}}} =  \frac{\sigma_{	{max}}^{k+1}}{1 - \sigma_{	{max}}}\\ 
.
\end{align*}


\subsection*{Question 2 :: Part k)}
So from part j) we have:
\[
    ||A^* - \sum_{n=0}^{k} A^n||_2 = \frac{\sigma_{\text{max}}(A)^{k+1}}{1- \sigma_{\text{max}}(A)}
.\]
Recall that $f(k) = O(\phi(k))$ as $k\to \infty$ means $\left| f(k) / \phi(k) \right| $ is bounded
for sufficiently large $k$. So clearly  $\phi(k) = \sigma_{\text{max}}(A)^k$ and we have
\[
\left| \frac{f(k)}{\phi(k)} \right|  = \frac{\sigma_{\text{max}}(A)}{1 - \sigma_{\text{max}}(A)}
.\]
which is certainly bounded for all $k$.
So our big-O bound is  $O(\sigma_{\text{max}}(A)^{k})$ and we see
that our error gets exponentially small as k gets large so it is a good approximation.

\section*{Question 3}
See uploaded .ipynb.

\end{document}
